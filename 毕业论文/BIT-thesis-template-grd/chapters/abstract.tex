%%==================================================
%% abstract.tex for BIT Master Thesis
%% modified by yang yating
%% version: 0.1
%% last update: Dec 25th, 2016
%%==================================================

\begin{abstract}
空指针引用是程序中比较常见的缺陷之一,研究表明该缺陷在编译后的程序中大量存在,因此给软件的稳定性带来很大威胁。出于对检测效率和精度的平衡,现有工具在工作原理和检测范围等方面各不相同,无法全面检测该类缺陷,大量的误报也降低了此类工具的实用价值。

本文设计了一个基于SonarQube平台运行的插件——BIT-Detector,它集成多种静态代码分析工具对代码进行检测,然后把不同工具产生的检测报告进行交叉验证,将检测出的缺陷按照可信度优先级进行排序以提升检测报告的可信度。结果表明所有工具能够同时检测的缺陷可信度非常高,但是那些只被部分工具报告出的缺陷的真实性难以判定。为了提升BIT-Detector的能力,本文提出了利用深度学习技术构建模型判定不同工具检测结果可信度的方法。

为了构建训练数据集,本文通过对一些开源代码进行语法分析并在合适的地方对其语法结构进行改造来生成大量空指针引用缺陷用例。然后生成测试用例的控制流图,并对图进行了适当的改造和压缩,最后从八个维度抽取出图中每个结点的代码特征。另外,为了便于模型训练,本文将不同工具实际检测结果的准确性作为每个训练数据的标签。

利用生成的数据集,本文设计了图特征抽取模型和特征分类模型。前者可以将包含代码特征信息的控制流图转换为一定维度的向量,后者可以根据标签(不同工具检测结果的准确性)对这些向量进行分类。通过这两个模型,可以评估目标代码在不同工具下检测结果的置信度。

实验结果表明,相对于单一的工具检测,BIT-Detect采用的交叉验证和缺陷可信度排序的方式大大提升了报告的实用性,某种程度上显著地降低了误报率。对于真实性难以判定的缺陷,本文设计的深度学习模型可以给出不同工具检测结果的置信度来参与决策,从而进一步提升了缺陷排序的准确性。

\keywords{空指针引用缺陷;静态检测;深度学习;交叉验证;}
\end{abstract}

\begin{englishabstract}

   Null pointer dereference is one of the common defects in the program. Research shows that this defect exists in a large number of compiled programs, therefore poses a great threat to the stability of software. Due to the balance between efficiency and accuracy of detection, the existing tools can not fully detect such defects as which vary in terms of operational principle and range of detection. A large number of false positives also reduce the practical value of such tools.
   
   This article designs a plug-in based on the SonarQube platform, BIT-Detector, which integrates multiple static code analysis tools to detect the code, then cross validates the detection reports generated by different tools, and sorts the defects by reliability to increase credibility of the detection report. The results show that the reliability of defects detected by all tools at the same time is very high, but the veracity of defects reported only by some tools is difficult to determine. In order to improve the capability of BIT-Detector, this paper proposes a method of constructing models using deep learning techniques to determine the credibility of different tools.
   
   For building the training data set, this paper generates a large number cases of null pointer deference by parsing some open source code and adapting its grammar structure properly. Then the control flow graph of the test case is generated, and the graph is modified and compressed appropriately. Finally, the code features of each node in the graph are extracted from eight dimensions. In addition, for facilitating the training of the model, the accuracy of actual detection results of different tools is used as a label for each training data.
   
   Using the generated data sets, this article designs a graph feature extraction model and a feature classification model. The former can convert control flow graphs containing code feature information into vectors, and the latter can classify these vectors according to the labels (accuracy of the detection results of different tools). Through these two models, the confidence level of the report from detecting target code with different tools can be evaluated.
   
   The experimental results show that compared to a single tool detection, the BIT-Detect's use of cross-validation and defect reliability ranking method greatly enhances the practicality of the report and reduces the false positive rate to some extent. For the defects whose authenticity is difficult to judge, the deep learning model designed in this paper can give the confidence of detection results of different tools to help make decision, thus further improving the accuracy of defect ranking.
   
\englishkeywords{null pointer dereference; static detection; deep learning; cross validation;}

\end{englishabstract}

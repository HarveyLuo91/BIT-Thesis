%%==================================================
%% conclusion.tex for BIT Master Thesis
%% modified by yang yating
%% version: 0.1
%% last update: Dec 25th, 2016
%%==================================================


\begin{conclusion}

随着软件规模的增长,代码安全问题日益突出。在众多的缺陷类型中,空指针引用缺陷出现的比例较高,且难以被检测,而其一旦出现将会带来巨大的损失。因此空指针引用缺陷的检测问题一直都是研究人员关注的热点。到目前为止,代码静态检测领域涌现出了大量的工具用以检测空指针引用等缺陷问题,但是研究表明这些工具的检测报告包含了大量的误报,导致了实用性的不足。而由于静态代码检测技术的限制,各种工具在权衡检测准确性和全面性方面做了不同取舍,因而导致不同工具产生的缺陷报告具有较大差别。

本文提出了一种交叉验证的思路,综合不同工具的检测报告来提升代码检测结果的精确率和覆盖率。并据此开发了基于SonarQube平台运行的插件——BIT-Detector。该插件可以在软件的持续集成环境下同时使用多种工具对代码进行检测,并且能够有效综合不同工具产出的缺陷报告,根据缺陷的可信度进行排序,从而提升单一工具在代码检测上的不足之处。其中,对于部分缺陷的置信度评估是本文研究的重点,本文利用深度学习方法,在构建了大量空指针引用缺陷用例的基础上,训练出图的向量化模型和依据工具检测能力的代码分类模型,其中进行了大量工作来建模缺陷代码的整体特征。

本文主要工作成果有下面四点:

(1)提出交叉验证现有静态检测工具的思想,成功实现基于SonarQube平台的插件BIT-Detector,该工具利用缺陷可信度优先级排序的方式可以大幅提升检测报告的实用性。

(2)提出一种批量生成缺陷代码用例的方法,利用AST工具解析目标代码的抽象语法树,在合适的位置适当修改代码的语法结构,利用脚本自动化编译,执行并验证生成效果。依据此方法成功利用现有开源代码生成了大量空指针引用缺陷用例。

(3)建模了空指针缺陷用例的整体特征。本文利用Soot工具构建空指针引用缺陷代码的控制流图,并结合过程间调用图生成程序的全局控制流图,继而对该图进行合理压缩,在基本块的层次上提取合理结点语义特征,使用包含语义特征的全局控制流图来表示一段空指针引用缺陷程序。

(4)利用深度神经网络将代码的结构特征和语义特征向量化,并构建学习模型对缺陷代码依据工具的检测效果进行分类,从而评估多工具检测报告中出现的结果不一致的缺陷的真伪性。

本文仅仅是探讨对Java代码中出现的空指针引用缺陷的检测,理论上本文使用的方法可以应用于任何代码缺陷和任意代码语言。未来的工作除了继续完善本工具的集成,性能和稳定性,也将会尝试扩展本文使用的方法,完成对更多缺陷类型的检测和更多语言的支持。

\end{conclusion}
%%==================================================
%% thanks.tex for BIT Master Thesis
%% modified by yang yating
%% version: 0.1
%% last update: Dec 25th, 2016
%%==================================================

\begin{thanks}
行文至此,我的硕士生涯也就像这篇毕业论文一样走到了尾声。

在北理的两年生活,虽然还是习惯了实验室,食堂和宿舍的三点一线,但是已经感觉到自己从适应学校到适应社会的转变。参加了一些学术会议,见识了很多大牛,慢慢懂得了对待学术应该更加谦卑。参加了一些社会实践,感受了职场生活,也慢慢懂得了对待工作应该更加专业。更重要的,慢慢懂得了应试能力和学习能力的区别,明白了生活是一道主观题而不是客观题,学会了去承担责任,做出取舍。

面对不断逼近的六月,离别也如期而至。解锁人生新关卡带来的激动和喜悦很快便被不舍和失落淹没。在跟过去道别之前,首先要感谢我的导师田东海老师和计卫星老师,他们在本论文的选题和研究上给予了重要的帮助。学贵得师,亦贵得友,两位老师谦和严谨而又不失活泼的处事风格无论在学术上还是生活上都深深地感染着我,能够遇到两位老师是我这两年来的最大幸事。

然后感谢实验室的小伙伴:杨恬,高建花,谈兆年,张莎莎,他们对本文的顺利完成也提供了必要的帮助。还有其他在实验室一路陪伴我的兄弟姐妹:石剑君,田泽民,李安民,以及已经毕业的郑兴生,付文飞,高志伟,廖心怡,张露露,张晶晶。我会记得永远充满欢声笑语的931,一起讨论问题抑或欣赏电影的930,一起挥洒过汗水的羽毛球场,一起品尝过的美味大餐,以及所有一起爬过的山,走过的路。

特别感谢我的父母,感谢他们在背后默默的支持,平凡而伟大的亲情永远是我无畏前行的后盾!

还要感谢离我而去的人,是你让我懂得了怎么去爱和珍惜!

最后替十年后的我感谢现在一直努力的自己!

所有BIT的朋友,山高水长,江湖再见!








\end{thanks}

%%==================================================
%% thanks.tex for BIT Master Thesis
%% modified by yang yating
%% version: 0.1
%% last update: Dec 25th, 2016
%%==================================================

\begin{thanks}

马上就要毕业了,数十年的求学生涯,也即将画上句号。想到这里,不免思绪万千。

回想短短两年半的研究生生活,我的导师,计卫星老师,给了我很大的帮助,从一开始对研究领域的一无所知,到定下研究方向,再到完成论文的期间,计老师与我一起讨论了无数次,给了我非常多的指导与启发。计老师是一位学风严谨,认真负责的老师。每周,计老师都会安排组会,组会上,同学们会分享自己近期研究的内容,计老师会指导每一位同学的学习进度。如果同学有问题找计老师,计老师一定会很认真负责地跟同学一起讨论,来解决问题。在计老师的帮助下,我才能顺利度过研究生生涯,顺利完成这篇论文,在这里,对我的导师计卫星老师表示衷心的感谢与深深的敬意。

然后,我要感谢实验室的各位老师,包括石峰老师,高玉金老师,王一拙老师和卫晋老师,各位老师在我开题答辩的时候,给了我很好的意见与建议,使我能在写论文初期,意识到论文的一些问题,及时改正。要感谢实验室的石剑君师姐,张晶晶同学,廖心怡同学,高志伟同学,罗辉师弟,田泽明师弟,高建花师妹,杨恬师弟,李安民师弟与谈兆年师弟,以及我的舍友们,我们一起去打球,一起出去游玩,一起去吃大餐,一起度过了很多快乐的时光。感谢来了暖气的实验室,让我可以好好写论文。

最后,我要特别感谢我的父母,我的家人,他们虽然不懂我的学业上的问题,但是很多很难熬的时候,我的父母都在我的身边关心我,开导我,帮我度过难关,是我坚强的后盾,是我的港湾。我要感谢我的朋友们,特别是一起在北京的高中同学们,我们一起度过了很多快乐的时光,他们给了我很多很好的建议,给了我很多鼓励。
\end{thanks}

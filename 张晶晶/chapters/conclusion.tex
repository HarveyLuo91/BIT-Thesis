%%==================================================
%% conclusion.tex for BIT Master Thesis
%% modified by yang yating
%% version: 0.1
%% last update: Dec 25th, 2016
%%==================================================


\begin{conclusion}

随着大数据时代的到来,信息技术与互联网的快速发展,数据量呈爆炸式增长。数据中包含着巨大的价值,而如何从海量的数据中挖掘出对人类有价值的信息,是现代社会的一个巨大的挑战与机遇。数据的压缩与异常数据检测,是在数据的分析与挖掘模型中重要的两个环节,在工业与生活中,也具有广泛的应用。流式数据是大数据中重要的一种形式。在对流式数据的处理与分析中,存在着以下几个问题。首先,是数据的存储问题,流式数据的数据量巨大,已有的存储介质不能存储所有的原始数据,所以,一些适用于静态数据集的挖掘方法无法直接使用在流式数据上。然后,是计算的实时性问题,流式数据是快速而且持续地产生的,计算的实时性尤其重要,如果数据处理不及时,很有可能发生数据的堆积,从而造成数据丢失。最后,流式数据往往是动态变化的,数据的分布总在变化,这使得很多静态数据集上研究的算法无法适应,效率与性能大大降低。上述的这些问题,给流式数据的研究带来了困难与挑战。基于上述问题,本文主要对流式数据的数据压缩与异常数据检测的加速进行了研究,使其能够快速计算,保证流式数据处理的实时性。本文的主要研究成果可以概括如下:

一,对数据的异常数据检测与数据的压缩进行了研究与总结。总结介绍了异常数据的定义,异常数据是不符合正常行为模式定义的数据模式,异常数据分为点异常,上下文异常和集体异常。而且异常数据检测的输出有分数型和标记型两种形式。然后又介绍了异常数据检测的相关算法,分析了各种算法的优缺点与使用范围。介绍了数据压缩的意义和现有的数据压缩算法,包括分段表示法、频域法、奇异值分解法与符号表示法。分析了各种算法的优缺点与使用范围,并且阐述了分段表示法最常使用的原因。

二,在对流式数据进行异常数据检测时,针对流式数据数据量巨大的特点,提出了改进的增量LOF算法。第一,将其空间划分为多个网格,将流式数据的数据点映射到网格中,可以解决流式数据数据量大,无法全部存储的问题。第二,设计网格的特征向量,将网格中的有权值的中心点代替映射到网格中的所有数据点,来进行增量的LOF算法的检测,可以减少计算量,加速检测异常数据速度,保证流式计算的实时性。第三,实验表明,该算法不但可以有效检测出异常数据,而且检测异常数据的速度更快,效率更高。

三,在对流式数据进行压缩时,选择简单直观常用的分段多项式拟合算法,提出了加速算法。第一,针对分段多项式拟合,给出了最小二乘法的解决过程。第二,针对静态时序数据的压缩,分别给出了平均分段与不平均分段的加速方法,通过建立缓存,来直接使用之前的计算结果,减少矩阵计算,加快了计算的速度。并且分别针对周期时间采集的与非周期时间采集的时序数据,给出了不同的加速方式。第三,针对流式时序数据,给出了使用滑动窗口算法的压缩过程,而且对于周期时间采样的时序数据,根据其时刻序号的特点,给出了使用缓存减少计算量的方法,加速压缩过程。针对非周期时间采样的时序数据,因为采样间隔不确定,所以不再使用空间替换时间的方法,而提出一种增量计算的方法,减少计算量和窗口内的数据点的存储量,提高了计算效率。

\end{conclusion}